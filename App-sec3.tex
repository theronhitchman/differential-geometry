\documentclass[Shifrin_Solutions_Spring_2015]{subfiles}
\begin{document}


\section{Differential Equations}

\begin{exercise}Suppose that $M(s)$ is a differentiable $3\times 3$ matrix function of $s$, $K(s)$ is a skew-symmetric $3\times 3$ matrix function of $s$, and
\[
M'(s) = M(s)K(s), \qquad M(0)=0 .
\]
Show that $M(s) = 0$ for all $s$ by showing that the trace of $(M^TM)'(s)$ is identically $0$.
\end{exercise}

\begin{proof}[Solution]
We note first that
\[
\begin{split}
(M^TM)' & = (M')^T M + M^T M' \\
	& = K^T M^T M + M^T M K \\
	& = - K (M^TM) + (M^TM)K .
\end{split}
\]
We then deduce that the trace of $(M^TM)'$ is zero because the trace of a commutator is always zero. Since taking a trace is a linear operation, we know that
\[
\tr((M^T M)') = (\tr(M^T M))' .
\]
So we see that the function $s \mapsto \tr(M(s)^TM(s))$ has initial value $0$ and has derivative equal to $0$ for all $s$. Thus this trace is always zero.

By construction we know that $M^TM$ is always a symmetric matrix. In fact, if we write $M = \begin{pmatrix} v_1 & v_2 & v_3 \end{pmatrix}$ as a matrix of column vectors, then a simple direct computation shows
\[
\tr(M^T M) = ||v_1||^2 + ||v_2||^2 + ||v_3||^2 .
\]
Since we have already seen that this quantity is always zero, we deduce that $v_1 = v_2  = v_3 = 0$. This means that $M = 0$.
\end{proof}

\vspace{1cm}

%%%%%%%%%%%%%%%%%%%%%%%%

\begin{exercise}(Gronwall Inequality and Consequences)
\begin{itemize}
\item[a.] Suppose that $f:[a,b)\rightarrow \mathbb{R}$ is differentiable, non-negative, and $f(a) = c > 0$. Suppose that $g:[a,b) \rightarrow \mathbb{R}$ is continuous and $f'(t) \leq g(t)f(t)$ for all $t$. Prove that
\[
f(t) \leq c \exp\left(\int_a^t g(u)\, du\right) \text{ for all $t$.}
\]

\item[b.] Conclude that if $f(a) = 0$, then $f(t) = 0$ for all $t$.

\item[c.] Suppose now $v:[a,b)\rightarrow \mathbb{R}^n$ is a differentiable vector function, and $M(t)$ is a continuous $n\times n$ matrix function for $t \in [a,b)$, and $v'(t) = M(t)v(t)$. Apply the result of part (b) to conclude that if $v(a) = 0$, then $v(t) = 0$ for all $t$. Deduce uniqueness of solutions to \emph{linear} first order differential equations for vector functions.

\item[d.] Use part (c) to deduce uniqueness of solutions to linear $n$th order differential equations.

\end{itemize}
\end{exercise}

\begin{proof}[Solution]
For part (a), note that $f'/f \leq g$ for all $t$. We integrate this relationship to see that
\[
\ln ( f(t) ) - \ln (f(a)) = \int_a^t \dfrac{f'(u)}{f(u)}\, du \leq \int_a^t g(u) \, du .
\]
The desired result follows from a simple manipulation.

For part (b), assume that $f(a) = 0$.  The proof used in the last part is insufficient to get this part done, even though it is very straightforward. Let us give another proof that handles both cases. We consider the function $h(t) = \exp\left(\int_a^t g(u)\, du\right)$. Note that $h(a) = 1$, and $h'(t) = g(t)h(t)$. So now
\[
\dfrac{d}{dt}\left( \dfrac{f(t)}{h(t)}\right) = \dfrac{h(t) f'(t) - f(t) h'(t)}{[h(t)]^2} \leq \dfrac{h(t)g(t)f(t) - f(t)g(t)h(t)}{[h(t)^2]} = 0.
\]
This means that $f/h$ is a non-increasing function. So, for all $t>a$, we have
\[
\dfrac{f(t)}{h(t)} \leq \dfrac{f(a)}{h(a)} = f(a).
\]
This translates directly to $f(t) \leq f(a) \int_a^t g(u)\, du $. Clearly, now if $f(a) = 0$, then the non-negative function $f$ must vanish. Note that here we only need the function $h$ to not vanish.

Alternately, one could try to finish this part by considering the function $\tilde{f} = f + c$. Since $f$ is non-negative, we have that
$\tilde{f}'(t) = f'(t) \leq f(t) g(t) \leq \tilde{f}(t)g(t)$. But this requires that $g$ does not take negative values.


For part (c), follow the hint and define $f(t) = ||v(t) ||^2$ and $g(t) = 2n \max\{|m_{ij}(t)|\}$.
Let us first do the computation we will need, and then put it into context. By a careful use of the Cauchy-Schwarz inequality, we have that
\[
\begin{split}
f'(t) & = \dfrac{d}{dt}||v(t)||^2 = 2v(t)\cdot v'(t) = 2 v(t) \cdot M(t) v(t) \\
	& = 2 \sum_{i=1}^n v_i \sum_{j=1}^n m_{ij}v_j \\
	& \leq 2 \max\{|m_{ij}(t)|\} \left(\sum_{i=1}^n v_i \right)^2 \\
	& \leq 2 \max\{|m_{ij}(t)|\} n \sum_{i=1}^n v_i^2  \qquad \mbox{(here is C-S!)} \\
	& = 2n \max\{|m_{ij}(t)|\} ||v||^2 \\
	& = f(t)g(t).
\end{split}
\]
So, by part (b), if $v(a)=0$, then $v(t) \equiv 0$.

Now, suppose that $v$ and $w$ are two solutions to the linear iinitial value problem $x' = A x$, $x(0)=x_0$. Then their difference $u = v - w$ is a solution to the initial value problem $x' = Ax$, $x(0) = 0$. This falls into the situation we just discussed! We conclude that $u = v-w$ is identically zero, hence $v=w$. this means that solutions to the initial value problem are unique.


For part (d), we note that a linear ODE $n$th order ODE can be changed into a linear first order ODE by adding dummy variables. An example suffices:

Consider the third order system
\begin{align*}
x'' + 2x' +x  & = 0 \\
y''' + x' +y & = x'' .
\end{align*}
We introduce new variables $x_0 = x$, $x_1 = x'$, $x_2 = x''$, $y_0 = y$, $y_1=y'$, $y_2 = y''$ $y_3 = y'''$. Then our system can be rewritten as
\begin{align*}
x_0' & = x_1 \\
x_1' & = x_2 \\
x_2' & = -x_0 - 2x_1 \\
y_0' & = y_1 \\
y_1' & = y_2 \\
y_2' & = y_3 \\
y_3' & = x_2 - x_1 - y_0 \\
\end{align*}
which is now a first order system.

So, if the solutions to a first order system are unique, then the solutions to linear $n$th order systems must also be unique.


\end{proof}

\end{document}










































