\documentclass[Shifrin_Solutions_Spring_2018]{subfiles}
\begin{document}


\section{Parametrized Surfaces and the First Fundamental Form}

\begin{exercise}
Derive the formula given in Example 1(e) for the parametrization of the unit sphere.
\end{exercise}




\begin{exercise}
Suppose $\alpha(t) = x(u(t), v(t) )$, $a \leq t \leq b$, is a parametrized curve on a surface $M$. Show that
\[
\begin{split}
\mathrm{length}(\alpha) & = \int_a^b \sqrt{ \mathrm{I}_{\alpha(t)} ( \alpha'(t), \alpha'(t) ) \, } \, dt \\
	& = \int_a^b \sqrt{ E(u(t), v(t)) (u'(t))^2 + 2 F(u(t),v(t)) u'(t) v'(t) + G(u(t),v(t)) (v'(t))^2 \,   } \, dt .
\end{split}
\]
Conclude that if $\alpha \subset M$ and $\alpha^* \subset M^*$ are corresponding paths in locally isometric surfaces, then $\mathrm{length}(\alpha) = \mathrm{length}(\alpha^*)$.
\end{exercise}

\begin{proof}[Solution] By definition, the length of $\alpha$ as a space curve is
\[
\mathrm{length}(\alpha) = \int_a^b ||\alpha'(t) || \, dt
\]
The chain rule says that $\alpha'(t) = x_u u' + x_v v'$, so the integrand is
\[
\begin{split}
||\alpha'(t) || & = \alpha'(t) \cdot \alpha'(t) = \left(x_u u' + x_v v'\right)\cdot \left(x_u u' + x_v v'\right)\\
	& = x_u\cdot x_u (u')^2 + 2 x_u\cdot x_v u' v' + x_v\cdot x_v (v')^2 \\
	& = E (u')^2 + 2F u'v' + G(v')^2.
\end{split}
\]
This gives the desired result by a simple substitution.
\end{proof}

\begin{exercise}
Compute $\mathrm{I}$ for the following parametrized surfaces.
\begin{itemize}
\item[a.] the sphere of radius $a$: $x(u,v) = a ( \sin u \cos v, \sin u \sin v, \cos u )$
\item[b.] the torus: $x(u,v) = ( (a+b\cos u) \cos v , (a + b \cos u) \sin v , b \sin u )$ , ($0<b < a$),
\item[c.] the helicoid: $x(u,v) = (u \cos v , u \sin v, bv )$
\item[d.] the catenoid: $ x(u,v) = a (\cosh u \cos v , \cosh u \sin v, u)$
\end{itemize}
\end{exercise}

\begin{proof}[Solution] We take the surfaces in the given order.
\begin{itemize}
\item[a.] The sphere of radius $a$:
\begin{align*}
x_u & = \left( a \cos(u)\cos(v) , a \sin(u)\sin(v) , a \cos(u) \right) \\
x_v & = \left( -a \sin(u)\sin(v) , a \sin(u)\cos(v), 0 \right) \\
E & = x_u\cdot x_u = a^2 \\
F & = x_u\cdot x_v = 0 \\
G & = x_v\cdot x_v = a^2\sin^2(u) \\
\mathrm{I} &= \begin{pmatrix} a^2 & 0 \\  0 & a\sin^2(u) \end{pmatrix}
\end{align*}

\item[b.] The Torus:
\begin{align*}
x_u & = \left(-b\sin(u)\cos(v) , -b\sin(u)\sin(v) , b\cos(u) \right) \\
x_v & = \left( -(a+b\cos(u))\sin(v) , (a+b\cos(u))\cos(v) , 0 \right) \\
E & = x_u\cdot x_u = b^2 \\
F & = x_u\cdot x_v =  0 \\
G & = x_v\cdot x_v = \left(a+b\cos(u) \right)^2 \\
\mathrm{I} & = \begin{pmatrix}  b^2 & 0 \\ 0& \left(a+b\cos(u) \right)^2\end{pmatrix}
\end{align*}

\item[c.] The Helicoid:
\begin{align*}
x_u & = \left(\cos(v), \sin(v), 0 \right) \\
x_v & = \left( -u\sin(v) , u\cos(v) , b\right) \\
E & = x_u\cdot x_u = 1 \\
F & = x_u\cdot x_v = 0 \\
G & = x_v\cdot x_v = u^2+b^2 \\
\mathrm{I} &= \begin{pmatrix} 1 & 0 \\ 0 &  u^2+b^2 \end{pmatrix}
\end{align*}

\item[d.] The Catenoid:
\begin{align*}
x_u & = \left( a\sinh(u)\cos(v), a\sinh(u)\sin(v), a \right) \\
x_v & = \left( -a\cosh(v)\sin(v), a\cosh(u)\cos(v), 0\right) \\
E & = x_u\cdot x_u = a^2 \cosh^2(u) \\
F & = x_u\cdot x_v = 0 \\
G & = x_v\cdot x_v =  a^2 \cosh^2(u)  \\
\mathrm{I} &= a^2 \cosh^2(u)  \begin{pmatrix} 1 & 0 \\ 0 &  1 \end{pmatrix}
\end{align*}
\end{itemize}
\end{proof}

\begin{exercise}
Compute the surface area for the following parametrized surfaces.
\begin{itemize}
\item[a.] the torus: $x(u,v) = ( (a+b\cos u) \cos v , (a + b \cos u) \sin v , b \sin u )$ , where $0<b < a$, and  $ 0 \leq u, v, \leq 2\pi$
\item[b.] a portion of the helicoid:  $x(u,v) = (u \cos v , u \sin v, bv )$ , $1< u < 3, 0\leq v \leq 2\pi$
\item[c.] a zone of a sphere: $x(u,v) = a (\sin u \cos v , \sin u \sin v , \cos u )$, $0\leq u_0 \leq u \leq u_1 \leq \pi, 0 \leq v \leq 2\pi$
\end{itemize}
\end{exercise}

\begin{proof}[Solution] We work each example in turn, using the fact that $\mathrm{Area} = \int_U \sqrt{EG-F^2}\, dudv$.
\begin{itemize}
\item[a.] The torus:
\[
\mathrm{Area} = \int_0^{2\pi}\int_0^{2\pi} \sqrt{b^2(a+b\cos(u))^2} \, dudv =2\pi \int_0^{2\pi} ab + b^2\cos(u) \, du = 4\pi^2 a b.
\]
\item[b.] The Helicoid for $1<u<3, 0\leq v \leq 2\pi$:
\[
\mathrm{Area} = \int_1^3\int_0^{2\pi}\sqrt{u^2+b^2} \, dvdu = 2\pi \int_1^3 \sqrt{u^2+b^2} \, du
\]
This integral begs for a messy trig substitution...I'm happy with it as it is for now.

\item[c.] The portion of the sphere for $0\leq u_0 \leq u \leq u_1 \leq \pi, 0 \leq v \leq 2\pi$:
\[
\mathrm{Area} = \int_{u_0}^{u_1} \int_0^{2\pi} \sqrt{a^4\sin^2(u)}\, dudv = 2\pi a^2 \left( \cos(u_0) - \cos(u_1) \right).
\]
Note that $\cos(u)$ is the $z$-coordinate for a point on the sphere, so this says that the area of the portion of the sphere really depends only on the distance between the two bounding planes, not on their exact location!
\end{itemize}
\end{proof}

\begin{exercise}
Show that if all the normal lines to a surface pass through a fixed point, then the surface is (a portion of) a sphere.
\end{exercise}

\begin{proof}[Solution] Suppose that $P$ is the point in question. Then there exists a function $\lambda=\lambda(u,v)$ such that $P = x + \lambda n$. Even better, we may write this as $x-P = \lambda n$ by switching the sign of lambda. Now, by taking derivatives in the $u$ and $v$ directions, we find
\[
x_u = \lambda n_u + \lambda_u n, \qquad x+v = \lambda n_v + \lambda_v n
\]
Recall that since $n$ is a unit vector, $n\cdot n_u =0$ and $n\cdot n_v = 0$. Now we can dot the two equations above with $n$ to find
\[
0 = x_u \cdot n = \lambda_u \qquad 0 = x_v \cdot n = \lambda_v.
\]
This means that $\lambda$ is a constant, and hence $||x-P|| = |\lambda|$ is a constant. This implies that $x$ is some portion of the sphere of radius $|\lambda|$ around $P$.
\end{proof}

\begin{exercise}
Check that the parametrization $x(u,v)$ is conformal if and only if $E=G$ and $F=0$.
\end{exercise}

\begin{proof}[Solution] First, let us look at a general set-up.  If $\theta$ is the angle between $v = \begin{pmatrix} a & b\end{pmatrix}$ and $w = \begin{pmatrix} c & d \end{pmatrix}$, then
\[
\cos\theta = \dfrac{ac+bd}{\sqrt{a^2+b^2}\sqrt{c^2+d^2}},
\]
and the angle $\theta'$ between their images $\hat{v} = ax_u + bx_v$ and $\hat{w} = c x_u + d x_v$ is defined by
\[
\cos \theta' = \dfrac{ac E + (ad+bc)F + bd G}{(a^2 E + 2ab F + b^2 G)^{-1/2} (c^2 E + 2cd F + d^2 G)^{-1/2} } .
\]


First suppose that $E=G$ and $F=0$. Then it is not hard to see that
\[
\cos \theta' = \dfrac{ac E +  bd E}{(a^2 E + b^2 E)^{-1/2} (c^2 E + d^2 E)^{-1/2} }  = \dfrac{ac+bd}{\sqrt{a^2+b^2}\sqrt{c^2+d^2}} = \cos\theta .
\]
So the angles agree and the parametrization is conformal.\\

Now suppose that the parametrization is conformal. If we use the orthogonal vectors $v=\begin{pmatrix} 1 & 1\end{pmatrix}$ and $w=\begin{pmatrix} 1 & -1\end{pmatrix}$, then
\[
0 = \cos\theta = \cos\theta' =\dfrac{E + (0)F - G}{( E + 2 F +  G)^{-1/2} ( E - 2 F +  G)^{-1/2} },
\]
which implies that $E=G$.

Similarly, if we choose the orthogonal vectors $v=\begin{pmatrix} 1 & 1\end{pmatrix}$ and $w=\begin{pmatrix} 1 & -1\end{pmatrix}$, then
\[
0 = \cos\theta = \cos\theta' =\dfrac{(0)E + F +(0) G}{( E )^{-1/2} (  G)^{-1/2} }= \dfrac{F}{\sqrt{EG}},
\]
so that $F=0$.
\end{proof}


\begin{exercise}
Check that a parametrization preserves area and is conformal if and only if it is a local isometry.
\end{exercise}

\begin{proof}[Solution] Recall that the area form is $\sqrt{EG-F^2}dudv$. \\

If the parametrization is a local isometry, then $E=G=1$ and $F=0$. Clearly the parametrization is conformal by the last exercise, and the area form on the surface is $dudv$, so areas are preserved. \\

Now, suppose that the parametrization is conformal. Then $E=G$ and $F=0$, so the area form is now $\sqrt{E^2}dudv = |E|dudv$. If areas are preserved, then the area form must be just $dudv$, so we deduce that $E= \pm 1$. If it is $+1$ we are done. Now note that since $I$ is positive definite, $E$ must be positive!
\end{proof}


\begin{exercise}
Check that the parametrization of the unit sphere by stereographic projection is conformal.
\end{exercise}

\begin{proof}[Solution] This follows from Exercise 6 by doing a computation. For the purposes of this exercise, let $\xi = u^2+v^2+1$.\\
Our parametrization is
\[
x(u,v) = \xi^{-1} \left( 2u , 2v , u^2 +v^2 -1 \right).
\]
Now we compute the first fundamental form directly. The tangent vectors are
\begin{align*}
x_u & = 2\xi^{-2} \left( -u^2+v^2+1 , -2uv , 2u\right) , \\
x_v &= 2\xi^{-2} \left( -2uv , u^2-v^2 +1 , 2v \right) .
\end{align*}
So the components of the first fundamental form are
\[
E  = 4\xi^{-2} , \quad F = 0 , \quad G = 4\xi^{-2} .
\]
Thus stereographic projection is conformal.
\end{proof}

\begin{exercise}
Project the unit sphere (except for the north and south poles) radially outward to the cylinder of radius $1$ by sending $(x,y,z)$ to $(x/\sqrt{x^2+y^2} , y/\sqrt{x^2+y^2} , z)$. Check that this map preserves area, but is neither a local isometry nor conformal.
\end{exercise}

\begin{proof}[Solution] Our parametrization of the sphere is
\[
x(u,v) = \left(\sin u \cos v , \sin u \sin v, \cos u \right).
\]
If we we make coordinates on our cylinder by $x^{\ast} = p\circ x$, where $p$ is the projection defined in the problem, then
\[
x^{\ast}(u,v) = \left( \cos v , \sin v, \cos u \right) .
\]
We have already computed the first fundamental form for the sphere parametrized by $x$. It is
\[
\mathrm{I} = \begin{pmatrix}
1 & 0 \\ 0 & \sin^2 u
\end{pmatrix}.
\]
In order to compare the geometries, we need to compute the first fundamental form for $x^{\ast}$.
\[
I^{\ast} = \begin{pmatrix}
\sin^2 u & 0 \\ 0 & 1
\end{pmatrix}
\]
Since the first fundamental forms at corresponding points disagree, we see that the projection is not a local isometry. But we do see that $\sqrt{EG-F^2} = \sqrt{E^{\ast}G^{\ast}-(F^{\ast})^2}$, so the map is area-preserving. Finally, to show that the projection is not conformal, we must find a pair of corresponding angles which are measured differently. We use the tangent vectors which are images of $a = \begin{pmatrix} 0 & 1\end{pmatrix}$ and $b= \begin{pmatrix} 1 & 1 \end{pmatrix}$. On the sphere, the images of these vectors have angle $\theta$ which is defined by
\[
\cos\theta = \dfrac{\sin^2 u}{\sqrt{\sin^2 u} \sqrt{1+\sin^2 u}}.
\]
But on the cylinder, the angle between them is $\theta^{\ast}$ defined by
\[
\cos\theta^{\ast} = \dfrac{1}{\sqrt{1}\sqrt{1+\sin^2 u}}.
\]
These are different, so the angles do not agree.
\end{proof}

\begin{exercise}
Consider the hyperboloid of one sheet, $M$, given by $x^2 +y^2-z^2 = 1$.
\begin{itemize}
\item[a.] Show that $x(u,v) = ( \cosh u \cos v , \cosh u \sin v , \sinh u )$ gives a parametrization of $M$ as a surface of revolution.

\item[b.] Find two parametrizations of $M$ as a ruled surface $\alpha(u) + v \beta(u)$.

\item[c.] Show that $x(u,v) = \left( \dfrac{uv+1}{uv-1}, \dfrac{u-v}{uv-1}, \dfrac{u+v}{uv-1} \right)$ gives a parametrization of $M$ where \emph{both} sets of parameter curves are rulings.
\end{itemize}
\end{exercise}

\begin{proof}[Solution] This is clearly in the form of a surface of revolution where the profile curve is the hyperbola in the plane $x=0$ defined by $\alpha(t) = (0 ,\cosh(t), \sinh(t))$.  Since $\alpha$ covers the entire hyperbola $x=0, \,  y^2-z^2=1$, our function $x(u,v)$ gives a parametrization of the entire hyperboloid.\\

To find parametrizations as a ruled surface, we must find the rulings. Let's cut the surface with planes through the origin and hope to find the rulings as intersections. Since the surface is rotationally symmetric about the $z$-axis, we need only look at planes of the form $x+az=0$ where $a$ is a constant. So, solving the set of simultaneous equations that results, we find:
\[
\left\{ \begin{array}{l} (a^2-1)z^2 + y^2 = 1 \\ x+az = 0 \end{array}\right.
\]
This points out the special role of $a^2=1$! So, if $a = \pm 1$, we see that this collapses onto the equations $y = \pm 1, \, x+az=0$. This is the intersection of two planes, and hence is a line.

If $a=1$, we get the line $t \mapsto (t,1,-t)= (0,1,0) + t(1,0,-1)$. If we rotate this around the $z$-axis, we obtain
\[
x(u,v) = \left( -\sin(u) + v\cos(u) , \cos(u) + v \sin(u) , -v\right),
\]
and if $a=-1$ , we get
\[
x(u,v) = \left( -\sin(u) + v\cos(u) , \cos(u) + v \sin(u) , v\right) .
\]
These are distinct since the slopes of the lines are different. \\

\todo[inline]{include pictures of the two rulings.}

\noindent\emph{Another way:} Looking at the surface, we see that the rulings must go through the ``waist'' circle, but not be contained in it. So we try to set things up that way. So, without loss of generality, suppose that our lines are defined by
\begin{align*}
x = &\cos\theta + a t \\
y = &\sin \theta + b t \\
z = &t
\end{align*}
where $a$ and $b$ are functions of $\theta$. These lines are supposed to live on our surface, so these coordinate functions must satisfy the algebraic equation defining the surface: $x^2+y^2-z^2 = 1$. We then find
\[
\begin{split}
1 & = x^2+y^2-z^2  = \cos^2\theta + (2a\cos\theta) t  + a^2 t^2 + \sin^2\theta + (2b\sin\theta) t + b^2 t^2 - t^2\\
	&= 1+ 2(a\cos\theta + b\sin\theta) t + (a^2+b^2-1)t^2.
\end{split}
\]
If this is to hold for all values of $t$, we must have that the coefficients vanish identically. Hence $a^2+b^2=1$ and $a\cos\theta +b\sin\theta =0$. By the first of these, we see that there is a number $\psi$ so that $a = \cos \psi$ and $b=\sin \psi$. If we put this into the second equation, we find that
\[
0 = \cos\theta\cos\psi + \sin\theta\sin\psi = \cos(\theta-\psi).
\]
From this, we deduce that $\theta-\psi = \pi/2$ or $3\pi/2$.

In the case where $\psi = \theta - \pi/2$, we see that $\cos\psi = \sin\theta$ and $\sin\psi = -\cos\theta$. Thus we get a realization of our surface like this:
\[
x(u,v) = \left( \cos\theta, \sin\theta , 0 \right) + v \left(\sin\theta , -\cos\theta, 1\right)
\]
In the case where $\psi = \theta - 3\pi/2$, a similar process leads us to
\[
x(u,v) = \left( \cos\theta, \sin\theta , 0 \right) + v \left(-\sin\theta , \cos\theta, 1\right).
\]
These are our two realizations as a ruled surface.\\

For the last part, we shall show that the $u$-curves are rulings. The result for the $v$-curves can be obtained in exactly the same way. So, fix $v= v_0$. Then the $u$-curve is
\[
\lambda(u) = \dfrac{1}{uv_0 -1}\left( uv_0 + 1 , u-v_0 , u+v_0\right).
\]
We must show that this traces a line in space, though it might not do so at a constant speed. Since a line with a reparametrization is of the form $\lambda(u) = w + f(u) v$ for some constant vectors $w, v$ and a scalar function $f(u)$, we can show that our curve is a line by showing that the difference between two values of the curve is always a scalar multiple of some constant vector. So we compute
\[
\lambda(u_1) - \lambda(u_2) = \dots = \dfrac{u_2-u_1}{(u_1v_0 -1)(u_2v_0-1)}\left( -2v_0 , v_0^2-1 , -v_0^2-1\right),
\]
which is the desired result.\\

\noindent\emph{Another Way}: Consider the $u$-curve as given above. One computes directly that
\[
\lambda'(u) = \dfrac{1}{(uv_0-1)^2} \left( -2v_0, v_0^2 - 1, -v_0^2-1\right).
\]
Therefore, if we let $f(u) = \int_0^u (tv_0-1)^{-2}\, dt$, we get
\[
\lambda(u) = C + f(u) W
\]
for constant vectors $C$ and $W$. So clearly this $u$-curve traces out points in a line.

\end{proof}

%%%%%%%%%%%%%%%%%%%%%%%%%%%%%%%%%%%%%%%%%%%%%%%%%%%%%%%%%%%%%%%%%%5


\begin{exercise}
Given a ruled surface $x(u,v) = \alpha(u) + v\beta(u)$ with $\alpha' \neq 0$ and $||\beta||=1$; suppose that $\alpha'(u), \beta(u)$, and $\beta'(u)$ are linearly dependent for every $u$. Prove that \emph{locally} one of the following must hold:
\begin{itemize}
\item[(i)] $\beta = \text{const}$;
\item[(ii)] there is a function $\lambda$ so that $\alpha(u) + \lambda(u) \beta(u) = \text{const}$;
\item[(iii)] there is a function $\lambda$ so that $(\alpha + \lambda \beta)'(u)$ is a non-zero multiple of $\beta(u)$ for every $u$.
\end{itemize}
Describe the surface in each of these cases.
\end{exercise}

\begin{proof}[Solution] If the vectors $\alpha, \beta$ and $\beta'$ are linearly dependent for every $u$, then there are functions $a, b$ and $c$ such that $a\alpha + b\beta + c\beta' =0$ for all $u$.
First, suppose that $a$ is always zero. Then we must have $b\beta+c\beta' = 0$. Since $\beta$ has norm one, we know that $\beta$ and $\beta'$ are orthogonal. The only way that these two vectors can be orthogonal and linearly dependent is if $\beta' = 0$. Therefore, $\beta$ is a constant. In this case, the ruled surface is a \emph{cylinder}.\\

Now suppose that $a$ is not zero. Then we may divide through by $a$, and by relabeling, assume that $a=1$. That is, $\alpha + b\beta +c\beta' = 0$. Let us rewrite this as
\[
0 = \alpha + c'\beta + c\beta' + (b-c')\beta = \left(\alpha + c\beta\right)' + (b-c')\beta.
\]
If $b=c'$, then we deduce that $\alpha+c\beta=P$ is a constant. In this case, the surface is \emph{conical}, where $P$ is the vertex of the cone. \\
If $b\neq c'$, then we see that $\gamma=\alpha+c\beta$ is proportional to $\beta$. In this case, we can rewrite out defining equation as $x = \gamma(u) + v \beta$, and the surface is a \emph{tangent developable}.
\end{proof}

%%%%%%%%%%%%%%%%%%%%%%%%%%%%%%%%%%%%%%%%%%%%%%%%%%%%%%%%%%%%%%%%%%%%%%%%55

\begin{exercise} We develop the Mercator parametrization. We want a parametrization $x(u,v)$ of the sphere, $u\in \mathbb{R}$, $v \in [0, 2\pi)$, so that the $u$-curves are longitudes and so that the parametrization is conformal. Letting $(\phi, \theta)$ be the usual spherical coordinates, write $\phi = f(u)$ and $\theta = v$. Show that the conformality and symmetry about the equator will dictate $f(u) = 2\arctan(e^{-u})$. Deduce that
\[
x(u,v) = ( \sech u \cos v , \sech u \sin v , \tanh u ) .
\]
\end{exercise}

\begin{exercise}

\end{exercise}

\begin{exercise}

\end{exercise}

\begin{exercise}

\end{exercise}

\begin{exercise}

\end{exercise}

\end{document}
