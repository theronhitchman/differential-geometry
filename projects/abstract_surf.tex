\documentclass[12pt]{amsart}
\usepackage{paralist}
\usepackage{fullpage}

\begin{document}
\title{Abstract Surfaces}
\author{Differential Geometry, Spring 2015}

\maketitle

\section*{Central Theme}

Explore Riemann's generalization of the concept of a surface to a type of geometric space that \textbf{need not be embedded in any ambient space}.

\section*{Minimum Requirements}

Write a paper exploring the basics of abstract Riemannian surfaces. 
\begin{compactitem}
\item 7-10 pages, in \LaTeX, with attention paid to standard English grammar, spelling and usage.
\item Give a clear definition of the concept of a \emph{manifold}, but in the case of dimension equal to 2. Pay particular attention to the way that overlap maps work, and think about why they are necessary.
\item Work out several examples, including at least these: the sphere, the torus.
\item Include images where appropriate.
\item Discuss how the geometry on an abstract surface can be defined by starting with the \emph{metric tensor} $g_{ij}$, which is a generalization of our first fundamental form.
\end{compactitem}

\section*{Extensions to Explore}

Have you ever really considered non-orientable surfaces?

\section*{Resources}

Just about any book on differential geometry at a level ``higher" than Shifrin will discuss the construction of abstract manifolds. Take a look at do Carmo's \emph{Differential Geometry of Curves and Surfaces}, or Milman and Parker's \emph{Elements of Differential Geometry} for a reasonable start.

If you are interested in Riemann's ground-breaking lecture, see Spivak's \emph{A Comprehensive Introduction to Differential Geometry, Vol. II.} 

\end{document}