\documentclass[12pt]{amsart}
\usepackage{paralist}
\usepackage{fullpage}

\begin{document}
\title{Bertrand Mates Project}
\author{Differential Geometry, Spring 2015}

\maketitle

\section*{Central Theme}

\emph{Bertrand mates} are a pair of space curves which have an interesting geometric 
relationship. They make pretty pictures, and give a good way to explore the details of the Frenet-Serret apparatus and its uses.

\section*{Minimum Requirements}

Write a paper exploring the basics of the Bertrand Mates. 
\begin{compactitem}
\item 7-10 pages, in \LaTeX, with attention paid to standard English grammar, spelling and usage.
\item Give a clear definition of Bertrand Mates.
\item Compute several examples, including at least these: a line, a circle, a circular helix, Viviani's curve (Struik example (3) on pages 9--10), the twisted cubic (Shifrin page 3).
\item Include images where appropriate.
\item Address as much of the following exercises as you can. (There is a lot of overlap here.) Organize the knowledge into a coherent exposition:
\begin{compactenum}
\item Shifrin $\S$1.2 \#19--22
\item Struik $\S$1.11 \#11--15.
\end{compactenum}
\end{compactitem}


\section*{Extensions to Explore}

There are no particular extensions of this project to deal with, as the exercises above are already pretty challenging. 

\section*{Resources}

Discussion of the theory of Bertrand mates is given in the exercises of both Shifrin $\S$1.2 and Struik $\S$1.11. There is also a reference in Struik to Bertrand's original paper. I am curious if Bertrand had a reason for studying these curves besides ``These are neat!''

\end{document}