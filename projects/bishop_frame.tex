\documentclass[12pt]{amsart}
\usepackage{paralist}
\usepackage{fullpage}
\usepackage{url}

\begin{document}
\title{Bishop Parallel Framing and Parallel Curves Project}
\author{Differential Geometry, Spring 2015}

\maketitle

\section*{Central Theme}

The Frenet-Serret framing along a curve only makes sense as long as the curvature $\kappa$ never vanishes. Points where $\kappa = 0$ make the normal direction undefined, and thus the binormal, too. To get around this hassle, Richard Bishop proposed a new set-up he called \emph{Relatively Parallel Adapted Frames}. This notion is related to the idea of having two parallel curves.

\section*{Minimum Requirements}

Write a paper exploring the basics of the Bishop's framings and parallel curves. 
\begin{compactitem}
\item 7-10 pages, in \LaTeX, with attention paid to standard English grammar, spelling and usage.
\item Give a clear definition of the Bishop framing.
\item Compute several examples of curves for which the Bishop framing might be a necessary thing.
\item Compute some examples of curves parallel to a given curve. Include at least a line, a circle, a circular helix, and the twisted cubic (Shifrin page 3).
\item Include images where appropriate.
\item Solve the following exercises out of Shifrin and weave them into a coherent story: $\S$1.2\#23, $\S$1.3\#9--11.
\end{compactitem}


\section*{Extensions to Explore}

Read through Bishop's paper and figure out the relationship between the functions $k_1, k_2$ from a Bishop framing and the standard $\kappa$ and $\tau$ from the Frenet-Serret Apparatus.
\section*{Resources}

The original paper by Richard Bishop appeared in the American Mathematical Monthly in 1975, and is available on jstor. If you are on campus, this url will work:

\url{http://www.jstor.org/stable/2319846}


\end{document}