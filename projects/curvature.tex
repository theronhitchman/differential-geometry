\documentclass[12pt]{amsart}
\usepackage{paralist}
\usepackage{fullpage}

\begin{document}
\title{Geometry of Gaussian Curvature}
\author{Differential Geometry, Spring 2015}

\maketitle

\section*{Central Theme}

Discuss truly geometric characterizations of the Gaussian curvature.

\section*{Minimum Requirements}

Write a paper exploring three characterizations of Gaussian curvature.
\begin{compactitem}
\item 7-10 pages, in \LaTeX, with attention paid to standard English grammar, spelling and usage.
\item Give a clear definition of Gaussian curvature
\item Use several examples including the sphere, the helicoid, and a cylinder.
\item Include images where appropriate.
\item Show that the Gaussian curvature is the ``infinitesimal area expansion factor" for the Gauss map.
\item Show that the Gaussian curvature helps measure the growth rate of the circumferences of small circles about a point.
\item show that the Gaussian curvature helps measure the growth rate of the area of little disks about a point.
\end{compactitem}

\section*{Resources}

Struik has some of this material near page 136.

Another reference is do Carmo's \emph{Differential Geometry of Curves and Surfaces} near pages 153 and 283.


\end{document}