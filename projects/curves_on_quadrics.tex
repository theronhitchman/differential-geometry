\documentclass[12pt]{amsart}
\usepackage{paralist}
\usepackage{fullpage}

\theoremstyle{plain}
\newtheorem*{theorem}{Theorem}

\begin{document}
\title{Curves on Quadric Surfaces Project}
\author{Differential Geometry, Spring 2015}

\maketitle

\section*{Central Theme}

There is a nice characterization (in terms of an ordinary differential equation) of when a space curve lies on some sphere. Can this be extended to the other classical quadric surfaces?

\section*{Minimum Requirements}

Write a paper exploring the basics of the curves lying on quadric surfaces. 
\begin{compactitem}
\item 7-10 pages, in \LaTeX, with attention paid to standard English grammar, spelling and usage.
\item Include images where appropriate.
\item Prove the theorem below.
\item Find analogs of these theorems for other quadric surfaces and prove them.
\item Include examples as appropriate. You might consider simple things like lines (for the hyperbolic paraboloid or the plane), and more complicated things like loxodromes or spherical helices on spheres.
\end{compactitem}

\begin{theorem} [Shifrin 1.2.18]
Let $\alpha$ be a unit-speed curve with non-vanishing curvature, $\kappa$, and torsion, $\tau$. Then there exists a sphere on which $\alpha$ lies if and only if
\[
\dfrac{\tau}{\kappa} + \left( \dfrac{1}{\tau} \left( \dfrac{1}{\kappa}\right)'\right)' = 0.
\]
\end{theorem}

\section*{Extensions to Explore}

How do the theorems you have found blend together? Is there a way to understand how the conditions change as you change the parameters which help define a family of quadric surfaces?

\section*{Resources}

There is a short discussion of the spherical case in Struik on page 32. There is related discussion in Shifrin in exercises $\S$1.2\#9(a) and $\S$1.2.24(c).

\end{document}