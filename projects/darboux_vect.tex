\documentclass[12pt]{amsart}
\usepackage{paralist}
\usepackage{fullpage}

\begin{document}
\title{The Darboux Vector and Areal Velocity Project}
\author{Differential Geometry, Spring 2015}

\maketitle

\section*{Central Theme}

There is a physical interpretation of the Frenet-Serret Apparatus due to Gaston Darboux which involves the notions of areal velocity and angular momentum. In this set-up, you consider the motion of the Frenet framing as somehow telling you about the motion of a rigid body in space. This is also referred to as a \emph{cinematical representation}.

\section*{Minimum Requirements}

Write a paper exploring the basics of the Darboux cinematical representation. 
\begin{compactitem}
\item 7-10 pages, in \LaTeX, with attention paid to standard English grammar, spelling and usage.
\item Give a clear definition of the Darboux vector, and discuss the proper interpretation.
\item What is areal velocity, and how is it relevant here?
\item Discuss the connection between the Frenet-Serret apparatus and the mechanics.
\item Compute several examples, including at least these: a line, a circle, a circular helix, Viviani's curve (Struik example (3) on pages 9--10), the twisted cubic (Shifrin page 3).
\item Include images where appropriate.
\item Show how to interpret the curvature, $\kappa$, and the torsion, $\tau$, in this set-up.
\end{compactitem}

\section*{Resources}

Struik mentions this material in exercise $\S$1.6\#18. The challenge in this project is that I don't have other resources to give you!

\end{document}