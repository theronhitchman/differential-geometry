\documentclass[12pt]{amsart}
\usepackage{paralist}
\usepackage{fullpage}

\begin{document}
\title{Fundamental Groups of Surfaces}
\author{Differential Geometry, Spring 2015}

\maketitle

\section*{Central Theme}

Explore some basic algebraic topology by starting with Poincar\'{e}'s notion of the \textbf{fundamental group} of a topological space.

\section*{Minimum Requirements}

Write a paper exploring the basics of fundamental groups. 
\begin{compactitem}
\item 7-10 pages, in \LaTeX, with attention paid to standard English grammar, spelling and usage.
\item Give a clear definition of the fundamental group of a surface.
\item Compute the fundamental group for some examples, including: a sphere, a torus, a cylinder, a genus 2 surface
\item Include images where appropriate.
\item Prove that the fundamental group is, in fact, a group by showing that concatentation of continuous paths is a well-defined, associative binary operation. Prove that a continuous function between two spaces will induce a group homomorphism between those two spaces.
\end{compactitem}


\section*{Extensions to Explore}

There is a tricky theorem called the Seifert-Van Kampen theorem...

\section*{Resources}

Armstrong's \emph{Basic Topology}, Massey's \emph{Algebraic Topology: An Introduction}, and especially Allen Hatcher's \ {Algebraic Topology} available from his web site for free as a pdf.

\end{document}