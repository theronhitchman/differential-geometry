\documentclass[12pt]{amsart}
\usepackage{paralist}
\usepackage{fullpage}

\begin{document}
\title{Geodesic Polar Coordinates}
\author{Differential Geometry, Spring 2015}

\maketitle

\section*{Central Theme}

Describe the way to make \textbf{geodesic polar coordinates} at a point on a surface. This is one way to make ``geometrically adapted'' coordinates, at least near one point.

\section*{Minimum Requirements}

Write a paper exploring the basics of geodesic polar coordinates. 
\begin{compactitem}
\item 7-10 pages, in \LaTeX, with attention paid to standard English grammar, spelling and usage.
\item Give a clear definition and construction of geodesic polar coordinates on a surface around a given point.
\item Compute several examples, including at least these: a sphere, a helicoid.
\item Include images where appropriate.
\item Be sure to include a reason why we know we can always make geodesic polar coordinates.
\end{compactitem}

\section*{Extensions}

There are a few neat ways to understand the Gaussian curvature if you have already set things up in geodesic polar coordinates. Explore those.

\section*{Resources}

Geodesic polar coordinates are discussed in most decent books on differential geometry at any level above that of Shifrin. Struik discusses them near page 136. The other place to look would be \emph{Differential Geometry of Curves and Surfaces} by do Carmo, page 286.
You can also find them in \emph{Elements of Differential Geometry} by Millman and Parker (page 176), but not necessarily in exactly polar form.

\end{document}