\documentclass[12pt]{amsart}
\usepackage{paralist}
\usepackage{fullpage}

\begin{document}
\title{Meusnier's Interpretation of Curvature}
\author{Differential Geometry, Spring 2015}

\maketitle

\section*{Central Theme}

Describe the geometry behind \textbf{Meusnier's Theorem}, which generalizes Euler's theorem on normal slices of a surface.

\section*{Minimum Requirements}

Write a paper exploring the basics of the Euler's theorem and Meusnier's theorem. 
\begin{compactitem}
\item 7-10 pages, in \LaTeX, with attention paid to standard English grammar, spelling and usage.
\item Give a clear statements of the two theorems, and proofs.
\item Compute several examples, including at least these: a sphere, a helicoid.
\item Give detailed descriptions of the way that these theorems give interpretations of 
curvatures of a surface in a particular direction.
\item Make clear and coherent pictures of the way that the normal slices are formed in each case, and show how the relevant curvatures are related to osculating circles.
\end{compactitem}

\section*{Resources}

Very spare versions of these results are in Shifrin as Propositions 2.3 and 2.5. So it can't hurt to start there. But there is much more that can be said. Struik has a better rendering of Meusnier's theorem (a stronger geometric statement than just the formula in Shifrin) on page 76, and a note about the relation to Euler's result on page 81.

You can find these results in many other books on classical differential geometry. For example, \emph{Elements of Differential Geometry} by Milman and Parker, \emph{Differential Geometry of Curves and Surfaces} by do Carmo, and \emph{A Comprehensive Introduction to Differential Geometry, Vol II} by Spivak.




\end{document}