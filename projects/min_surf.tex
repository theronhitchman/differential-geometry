\documentclass[12pt]{amsart}
\usepackage{paralist}
\usepackage{fullpage}

\begin{document}
\title{Minimal Surfaces}
\author{Differential Geometry, Spring 2015}

\maketitle

\section*{Central Theme}

Describe Minimal Surfaces and their connection with complex function theory.

\section*{Minimum Requirements}

Write a paper exploring the basics of minimal surfaces. 
\begin{compactitem}
\item 7-10 pages, in \LaTeX, with attention paid to standard English grammar, spelling and usage.
\item Give a clear definition of a minimal surface, and show how to find a set of isothermal coordinates on one. Use this to create a Weierstrass representation of the surface using complex functions.
\item Compute several examples.
\item Include images where appropriate.
\end{compactitem}


\section*{Extensions to Explore}

If you feel so inclined, look up how minimal surfaces get their name: as local area minimizers. This will involve some basic ``calculus of variations."

\section*{Resources}

Shifrin has a little bit about minimal surfaces in the last chapter. Struik is hiding some stuff, too, but it is pretty terse. You might also consult do Carmo's \emph{Differential Geometry of Curves and Surfaces}.

I have a copy of \emph{Six Themes on Variation} which has a chapter by Mike Wolf that might be of use.

\end{document}