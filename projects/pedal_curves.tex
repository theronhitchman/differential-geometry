\documentclass[12pt]{amsart}
\usepackage{paralist}
\usepackage{fullpage}
\usepackage{url}

\theoremstyle{plain}
\newtheorem*{theorem}{Theorem}

\begin{document}
\title{Pedal Curves Project}
\author{Differential Geometry, Spring 2015}

\maketitle

\section*{Central Theme}

This project is about a classical construction for planar curves called the 
\emph{pedal curve}. This should provide opportunity to explore interesting curves 
and their parametrizations. 

\section*{Minimum Requirements}

Write a paper exploring the basics of pedal curves. 
\begin{compactitem}
\item 7-10 pages, in \LaTeX, with attention paid to standard English grammar, spelling and usage.
\item Give a clear definition of the pedal curve associated to a given curve and a given point.
\item Compute several examples, including at least these: a line, a circle, the conchoid of Nicomedes, the folium of Descartes, and the figure eight curve.
\item Include images where appropriate.
\item For each example, give a parametrization of the curve and its pedal curve with respect to the origin.
\item Prove the theorem below.
\end{compactitem}

\begin{theorem} [Struik 1.13.15] Suppose that $\alpha$ is a given curve, and $\beta$ is the pedal curve of $\alpha$ with respect to a given point $A$. If $P$ is a point on $\alpha$ and $Q$ is the corresponding point on $\beta$, then $AQ$ makes the same angles with the pedal curve $\beta$ as $AP$ makes with $\alpha$.
\end{theorem}

\section*{Extensions to Explore}

Compute the pedal curves of the examples with respect to other points. How does the shape of the pedal curve change if the point changes?

\section*{Resources}

Struik makes explicit mention of pedal curves in the exercises in $\S$1.13.

There is a list of classical planar curves you might find helpful here:
\url{http://www-history.mcs.st-and.ac.uk/Curves/Curves.html}

\end{document}