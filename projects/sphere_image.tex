\documentclass[12pt]{amsart}
\usepackage{paralist}
\usepackage{fullpage}

\theoremstyle{plain}
\newtheorem*{theorem}{Theorem}

\begin{document}
\title{The Tangent Spherical Image}
\author{Differential Geometry, Spring 2015}

\maketitle

\section*{Central Theme}

The \emph{tangent spherical image} of a space curve is a way to visualize the moiton of a curve by using only the motion of the tangent vectors. It is also called the \emph{spherical indicatrix} or \emph{tangential indicatrix} of the curve. The point of this project is to get comfortable with space curves by exploring their tangent spherical images. 

\section*{Minimum Requirements}

Write a paper exploring the basics of the tangent spherical image. 
\begin{compactitem}
\item 7-10 pages, in \LaTeX, with attention paid to standard English grammar, spelling and usage.
\item Give a clear definition of the tangent spherical image.
\item Compute several examples, including at least these: a line, a circle, a circular helix, Viviani's curve (Struik example (3) on pages 9--10), the twisted cubic (Shifrin page 3).
\item Include images where appropriate.
\item Prove the two theorems below.
\end{compactitem}

\begin{theorem} [Struik 1.6.19] The ratio of the arclength element $ds_T$ of the tangential indicatrix to the arclength element $ds$ of the original curve is equal to the absolute value of the curvature of the given curve.
\end{theorem}

\begin{theorem} [Struik 1.11.9] The tangent spherical image of a curve is a circle if and only if the curve is a (generalized) helix.
\end{theorem}

\section*{Extensions to Explore}

The two theorems above are pretty strong bits of geometric information that can be gleaned from the tangent indicatrix. Consider one or more of these questions:
\begin{compactenum}
\item Is it possible to express the curvature and torsion of the tangential indicatrix 
in terms of the curvature and torsion of the original curve?
\item Is there anything interesting one can say by considering the \emph{normal indicatrix}?
\item Is there anything interesting one can say by considering the \emph{binormal indicatrix}?
\end{compactenum}

\section*{Resources}

There is a mention of this in Struik in the exercises at the end of $\S$1.6 and $\S$1.11. 

\end{document}