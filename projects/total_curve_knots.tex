\documentclass[12pt]{amsart}
\usepackage{paralist}
\usepackage{fullpage}
\usepackage{url}

\begin{document}
\title{The Total Curvature as a Measure of Knottedness}
\author{Differential Geometry, Spring 2015}

\maketitle

\section*{Central Theme}

There is a beautiful connection between the \emph{total curvature} of a closed space curve and the topology of that closed space curve when considered as a knot. The first step in this direction is a theorem due to Fenchel, which relies on the Cauchy-Crofton formula. This was later extended by F\'{a}ry and Milnor (independently). There is even some sense as to using total curvature to measure how complicated a knot is!

\section*{Minimum Requirements}

Write a paper exploring the basics of the relationship between the total curvature of a knot and its topology. 
\begin{compactitem}
\item 7-10 pages, in \LaTeX, with attention paid to standard English grammar, spelling and usage.
\item Give a clear definition of the total curvature of a space curve.
\item Compute several examples of knots of varying complexity.
\item Include images where appropriate.
\item Prove the theorems of Cauchy-Crofton, Fenchel, and F\'{a}ry-Milnor carefully and completely.
\end{compactitem}


\section*{Extensions to Explore}

Read the original literature by Milnor and F\'{a}ry (in translation) and explore how the 
total curvature of a knot measures the level of knottiness. Knot theory has advanced 
much since the 1950's, so maybe a bit of poking around on the internet will help settle 
things. You may run across the term \emph{bridge number}. Find and show us 
something cool.

\section*{Resources}

This material is newer than Struik's book! (Well, Fenchel's theorem gets a mention on page 204.) You can find some of the basics here in $\S$1.3 of Shifrin, and the exercises in that section, particularly Proposition 3.2, Theorem 3.3 and Theorem 3.4 and Exercise 11.

Here are three of the original papers, which are short and reasonably accessible:
\begin{compactenum}
\item A translation of F\'{a}ry's original paper into English:
\url{https://www.cs.duke.edu/~brittany/research/fary.pdf}
\item Milnor's first paper on the subject (jstor link should be available on campus)
\url{http://www.jstor.org/stable/1969467}
\item Milnor's second paper, exploring the deeper question
\url{http://www.mscand.dk/article/viewFile/10387/8408}
\end{compactenum}

\end{document}